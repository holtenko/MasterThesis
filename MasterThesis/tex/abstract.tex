%# -*- coding: utf-8-unix -*-
%%==================================================
%% abstract.tex for SJTU Master Thesis
%%==================================================

\begin{abstract}

上海交通大学是我国历史最悠久的高等学府之一,是教育部直属、教育部与上海市共建的全国重点大学,是国家 “七五”、“八五”重点建设和“211工程”、“985工程”的首批建设高校。经过115年的不懈努力,上海交通大学已经成为一所“综合性、研究型、国际化”的国内一流、国际知名大学,并正在向世界一流大学稳步迈进。 

十九世纪末,甲午战败,民族危难。中国近代著名实业家、教育家盛宣怀和一批有识之士秉持“自强首在储才,储才必先兴学”的信念,于1896年在上海创办了交通大学的前身——南洋公学。建校伊始,学校即坚持“求实学,务实业”的宗旨,以培养“第一等人才”为教育目标,精勤进取,笃行不倦,在二十世纪二三十年代已成为国内著名的高等学府,被誉为“东方MIT”。抗战时期,广大师生历尽艰难,移转租界,内迁重庆,坚持办学,不少学生投笔从戎,浴血沙场。解放前夕,广大师生积极投身民主革命,学校被誉为“民主堡垒”。

新中国成立初期,为配合国家经济建设的需要,学校调整出相当一部分优势专业、师资设备,支持国内兄弟院校的发展。五十年代中期,学校又响应国家建设大西北的号召,根据国务院决定,部分迁往西安,分为交通大学上海部分和西安部分。1959年3月两部分同时被列为全国重点大学,7月经国务院批准分别独立建制,交通大学上海部分启用“上海交通大学”校名。历经西迁、两地办学、独立办学等变迁,为构建新中国的高等教育体系,促进社会主义建设做出了重要贡献。六七十年代,学校先后归属国防科工委和六机部领导,积极投身国防人才培养和国防科研,为“两弹一星”和国防现代化做出了巨大贡献。

\keywords{\large 上海交大 \quad 饮水思源 \quad 爱国荣校}
\end{abstract}

\begin{englishabstract}

An imperial edict issued in 1896 by Emperor Guangxu, established Nanyang Public School in Shanghai. The normal school, school of foreign studies, middle school and a high school were established. Sheng Xuanhuai, the person responsible for proposing the idea to the emperor, became the first president and is regarded as the founder of the university.

During the 1930s, the university gained a reputation of nurturing top engineers. After the foundation of People's Republic, some faculties were transferred to other universities. A significant amount of its faculty were sent in 1956, by the national government, to Xi'an to help build up Xi'an Jiao Tong University in western China. Afterwards, the school was officially renamed Shanghai Jiao Tong University.

\englishkeywords{\large SJTU, master thesis, XeTeX/LaTeX template}
\end{englishabstract}

