%# -*- coding: utf-8-unix -*-
%%==================================================
%% abstract.tex for SJTU Master Thesis
%%==================================================

\begin{abstract}
农业物联网为农业生产环境的监控提供了新思路,可实现农业的智能化、信息化、精准化管理,是实现农业现代化的关键技术之一。基于农业物联网的智能温室系统,可通过网络监测温室环境,并根据作物的环境需求实施精准的温室控制,从而科学高效的管理温室。基于计算流体力学的温室仿真解决了复杂温室环境的建模分析问题,可以为智能温室系统的监测点位置优化和控制策略优化提供理论依据。

通过分析温室监控的特殊需求并参考物联网标准架构,本文提出了基于农业物联网的智能温室架构方案。根据该架构设计了智能温室监测与控制系统:感知控制层基于ZigBee和RS485传感器网络及计算机控制模块,并针对可靠性、可扩展性、灵活性和低功耗进行了优化设计;网络传输层支持多种数据传输方式和数据同步机制,并重点针对网络传输的安全性、可靠性和高效性进行了设计,建立了系统层间枢纽;应用层包含数据中心、WEB服务器和智能策略学习控制子系统,提供了基于Hadoop和MySQL的海量温室历史数据的云存储解决方案、高可用免维护的云服务器和基于CFD仿真模型的优化控制策略;终端接入层采用WEB前端技术和React Native为系统提供了可视化界面。

智能温室监测与控制系统在实验温室进行了实测运行,运行结果表明:无线传感器网络稳定可靠,温室内外数据采集准确正常,上传同步延迟低,控制指令下达准确迅速,控制状态获取正常。数据中心工作正常,未发生慢查询或异常。云端WEB服务运行稳定,接口响应迅速,数据返回完整准确。系统满足预期设计要求和实际生产要求。

针对南方地区典型的连栋塑料温室的夏季机械通风优化控制要求,建立了三维全尺度瞬态及稳态计算流体力学仿真模型。基于智能温室监控系统测量各种机械通风条件下的的温室内气温变化和分布,用实验验证了仿真模型瞬态和稳态计算的准确性和有效性。通过仿真模型模拟了室外高温条件下的风机数量、温室长度、入口温度及环境温度变化等参数对机械通风降温效果的影响程度,并模拟了不同数量风机启闭控制的降温效果。根据仿真分析结果优化了温室监测点位置方案和夏季机械通风控制策略。

经过对CFD仿真结果进行分析,优化后的温室监测点方案仅需最少5个监测点即可实现1280 $\text{m}^{2}$的温室整体环境状态的监测,有效减少了测点数量,降低了监测成本;优化后智能温室系统控制策略最高可减少约60\%的能源消耗,而植物冠层平均温度仅升高0.21℃,有效提高了机械通风降温效率,减少了能源消耗。

根据本文架构方案实现的智能温室监测与控制系统,节点布置灵活、监测范围广、运行功耗低、可靠性高、稳定性强、可灵活扩展,满足温室的智能和科学管理需求。结合CFD仿真分析,优化的监测点方案可有效降低监测成本,优化的机械通风控制策略可有效实现节能减排。系统还可实现异地温室集中互联,共同构建农业大数据和物联网平台,对农业科学研究和工程控制有重要意义,有助于提高农业现代化水平,推进农业物联网发展。

\keywords{\large 农业物联网 \quad 计算流体力学 \quad 温室 \quad 架构 \quad 监控系统}
\end{abstract}

\begin{englishabstract}

Agricultural Internet of Things provides a new idea for the monitoring and control of agricultural production environment. It can realize the intelligent, networked and precise management of agriculture. It is one of the key technologies to realize agricultural modernization. The intelligent greenhouse system based on agricultural Internet of things, can realize the monitoring  of the greenhouse environment through the network, and can implement accurate greenhouse control according to crop environmental requirements. Then, it can achieve the scientific and efficient management of greenhouse. Greenhouse simulation based on the CFD solves the problem of modeling and analysis of complex greenhouse environment, which can provide the theoretical basis for the optimization of sensors distribution and optimization of control strategy in intelligent greenhouse system.

By analyzing the special needs of greenhouse monitoring and reference to the Internet of Things standard architecture, this paper proposed a intelligent greenhouse architecture based on agricultural Internet of Things, and designed the intelligent greenhouse monitoring and control system based on this architecture. The sensing control layer is based on the ZigBee and RS485 sensor network and computer control module, and is optimized for reliability, scalability, flexibility and low power consumption. The network transport layer supports a variety of data transmission and synchronization mechanisms, and is optimized for security, reliability and efficiency. It establishes the hub of the system layers. The application layer includes data center, WEB server and intelligent strategy learning control subsystem. It provides a cloud storage solution based on Hadoop and MySQL for the massive greenhouse history data, and provides a high availability and maintenance-free cloud server and an optimal control strategy based on CFD simulation model. The terminal access layer using WEB front-end technology and React Native system provides a user interface.

The intelligent greenhouse monitoring and control system were deployed and tested in the experimental greenhouse. The result show that the system meets the expected design requirements and the actual production requirements. The wireless sensor network is stable and reliable, the data sampling is accurate and stable, the delay of synchronization is low, the control orders is issued quickly and accurately, the control status acquisition is accurate and stable. The data center is working fine, and there is no slow queries or exceptions. The cloud WEB services is running stably, the API responds quickly, the returned data is complete and accurate.

Aimed at the requirement of summer mechanical ventilation optimization for the typical multi-span plastic greenhouse in southern China, a full-scale 3-D transient and steady CFD model was established. By using the intelligent greenhouse monitoring system, the variation and distribution of temperature in the greenhouse under the various mechanical ventilation conditions were measured. The veracity and validity of the model for transient and steady simulation are verified by the comparison between the experiment and simulation results. Under the high ambient temperature, the CFD model was carried out to investigate the impact of design parameters, such as the number of working fans, the length of greenhouse, the inlet temperature and the ambient environment temperature, on the cooling effect of the mechanical ventilation. And then, the cooling effect under different control strategies of mechanical ventilation were simulated to reduce the energy consumption. According to the simulation results, the distribution of monitoring points and the summer mechanical ventilation control strategies were optimized.

After the analysis of the CFD simulation results, the system can monitor the whole greenhouse environment of 1280 $\text{m}^{2}$ with at least 5 monitoring points using the optimized distribution of monitoring points. It effectively reduce the number of monitoring points and the monitoring cost. And the optimized control strategy of the system can reduce the energy consumption by about 60\%, while the average temperature of plant canopy only increases by 0.21 ℃, which effectively improves the cooling efficiency of mechanical ventilation and reduces the energy consumption.

The intelligent greenhouse monitoring and control system based on the architecture of this paper has flexible node arrangement, wide monitoring range, low power consumption, high reliability, strong stability and flexible expansion, can meet the intelligent and scientific management requirements of the greenhouse. Combined with CFD simulation analysis, the optimized monitoring point distribution can effectively reduce the monitoring cost, and the optimized mechanical ventilation control strategy can effectively realize the energy saving and emission reduction. The system can also realize the centralized interconnection of the greenhouses in the different areas to build the agricultural big data and IoT platform. The system is of great significance to the agricultural scientific research and engineering control, and it is helpful to improve the level of agricultural modernization and promote the development of agricultural internet of things.

\englishkeywords{\large agricultural internet of things, Computational fluid dynamics, greenhouse, agriculture, monitoring and control system}
\end{englishabstract}

