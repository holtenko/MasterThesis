%# -*- coding: utf-8-unix -*-
%%==================================================
%% chapter1.tex for SJTU Master Thesis
%%==================================================

\chapter{总结与展望}
\label{chapter:Conclusion}

\section{研究工作总结}
本文根据一般物联网的架构设计和农业生产的特殊性,提出了一种基于农业物联网的智能温室架构设计,并基于该架构设计,对架构的每一层次进行了详细的设计和实现,最终完成了一套智能温室监测与控制系统,该系统可远程监测温室内的空气温湿度、土壤温湿度以及光照强度等信息,以及温室外的空气温湿度、风速、风向、大气压、太阳辐射强度以及降雨量等信息,可远程查看当前温室内的实时视频和图像,同时可以远程控制或智能控制温室内的风机、湿帘、顶窗、侧窗、遮阳网以及滴灌等设备。

建立了南方连栋塑料温室全尺度三维瞬态及稳态机械通风仿真模型,对温室机械通风过程进行分析,为温室设计提供优化的结构参数和通风设备配置,同时为智能温室监测与控制系统提供优化的监测点位置选择方案和机械通风控制策略。

该系统可以广泛应用于各种类型的温室,感知控制层具有通用性强、可靠性高、布置灵活、监测范围广等特点,并重点优化了节点的低功耗性能和无线传感器网络的稳定性和扩展性,自定义的网络通信流程实现灵活可控的采样周期,该层可以对温室环境实现全面感知。网络传输层具有接入灵活、安全、稳定、可靠等特点,可以兼容各种网络类型,搭建起物联网层间的通信桥梁。应用层扩展灵活、稳定性强、运行维护成本低,为系统内部实现了海量历史数据存储、数据分析处理和温室智能控制,并向外提供了丰富可靠的接口调用。终端接入层为农业生产管理人员、科研工作者和消费者提供可视化的温室远程监测和控制服务。系统实现了温室生产的物联网化、促进了温室的科学安全高效管理。

对本文的主要研究工作总结如下:
	\begin{enumerate}
		\item 基于农业物联网的智能温室架构设计与系统实现。结合物联网通用架构和农业生产的特殊环境,提出了一种基于农业物联网的智能温室架构方案设计,并对该架构的各个层次进行了分析定义和说明。基于提出的智能温室架构设计方案,对系统的感知控制层、网络传输层、应用层和终端接入层的硬件和软件进行了设计并实现。基于ZigBee和RS485的传感器网络实现温室环境的感知,优化了节点的低功耗性能,相比无低功耗设计的节点,功耗减少96.3\%,在3000 mAh电池供电的情况下可连续工作1~2个月,结合太阳能供电,满足温室感知层低功耗、长期无值守工作的需求。结合互联网和云服务技术,建立了高性能、高可用、安全稳定的网络传输和应用服务。利用大数据存储技术,提供了海量数据云端存储解决方案。运用前端技术开发功能丰富、界面美观、操作简单的可视化终端应用程序。系统构建完成后在上海崇明实验温室进行了实践应用,验证了架构方案的可行性与合理性,以及系统的有效性、稳定性和可靠性。
		\item 建立南方连栋塑料温室机械通风CFD仿真模型。针对典型南方连栋塑料温室的夏季机械通风建立三维全尺度瞬态及稳态计算流体动力学仿真模型,并通过基于智能温室监测与控制系统的温室机械通风实验,验证了仿真模型预测空气温度瞬态变化趋势和稳态分布的准确性。
		\item 温室监测点位置方案优化。基于对仿真结果的分析优化温室监测点位置方案,仅需最少5个监测点即可满足1280 $\text{m}^{2}$的连栋温室的监测,有效减少了监测点数量,降低了系统监测成本。
		\item 温室机械通风系统设计及机械通风控制策略优化。利用模型进行了室外高温条件下不同风机数量、温室长度、入口温度及环境温度变化的仿真模拟,并模拟了不同数量风机启闭控制策略的降温效果。仿真结果表明:开启风机的数量对降温速度影响较大,对稳定温度影响较小;与湿帘水温等直接相关的入口温度对稳定温度影响显著,每降低1℃,稳定温度平均可降低0.9℃;温室长度和外界环境温度对降温效果影响较小,温室长度增加50\%,平均稳定温度仅升高0.5℃;湿帘入口的不连续会严重影响温室部分区域的降温效果。控制策略研究表明:可合理控制风机开启时机、开启时长和风机数量,以达到节能减排,节约温室使用成本的目的。采用10台风机间歇运行和10台风机快速降温、4台风机维持的策略,可比10台风机持续运行分别降低约47\%和60\%的能源消耗,植物冠层平均温度仅升高0.94℃和0.21℃。本文建立的机械通风CFD模型可用于同类温室的优化设计和控制,对于其它类型温室的建模和优化具有参考意义。
	\end{enumerate}
\section{研究工作展望}
本文设计实现的基于农业物联网的智能温室监测与控制系统,虽然在实践应用中符合生产要求且效果良好,可有效提高温室生产过程的科学管理水平,但在研究过程中还存在许多可以改进的方向,主要集中在传感器种类全面化、网络技术升级、底层软件开发、控制策略机器学习实现、仿真模型优化等方面。
	\begin{enumerate}
		\item 传感器种类多样化、全面化。为了对温室环境进行更全面的监测,还需要在系统中添加更多种类的传感器,如土壤盐碱度、土壤养分、空气$\text{CO}_{2}$浓度等传感器。
		\item 网络技术升级。考虑使用6LoWPAN和ARM内核的单片机芯片,开发更加贴近物联网应用的低功耗传感器网络。
		\item 底层软件开发。采用更加底层的C语言代替Java开发效率更高的传感器数据采集程序。
		\item 控制策略机器学习实现。利用本文所实现的智能温室监测系统获取的海量温室历史数据,通过机器学习和数据挖掘方法,训练和优化温室控制策略,实现温室策略的自主学习更新。
		\item 仿真模型优化。在仿真模型中加入作物模型,分析种植作物的情况下温室内的环境分布和变化规律,同时可考虑种植不同作物,包括作物高度、作物种类等对温室环境的影响。还可加入更多的环境因素,如空气二氧化碳浓度等,进行更加接近真实环境的仿真模拟。
 	\end{enumerate}
