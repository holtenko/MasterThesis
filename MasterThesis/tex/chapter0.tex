%# -*- coding: utf-8-unix -*-
%%==================================================
%% chapter0.tex for SJTU Master Thesis
%%==================================================


\chapter{绪论}
\label{chapter:Introduction}

\section{研究背景和意义}
自古以来,我们中国的祖先就知道“民以食为天”“粮食定,天下定”这一治国理政之道\supercite{ChenYunWenXuan},中国作为一个人口大国,解决全国人民的粮食食物问题是关系到国计民生的根本问题。根据农业科学院预测到2050年我国的人口将超过15亿,人口的增加也必然带来粮食需求量的增加。随着我国经济的飞速发展和人们生活水平的提高,人们不再仅仅关注于温饱的问题,更加追求品质生活,更加关注食品的安全、营养和多样\supercite{FengZhiming2007,ZhaoQiguo2011}。随之而来的,农业生产在不断扩大总量的同时也需要不断升级产业结构,以提供更加安全和高品质农副产品。但是我国目前的农业生产的总体技术水平落后,现代化和信息化水平较低。农村的城镇化导致我国可用耕地面积不断减少,农村劳动力大量涌入城镇\supercite{YuJunli2001},务农人口急剧减少,农业生产人员逐步呈现老龄化和副业化\supercite{LuoChaobin2005},传统的农业生产模式已经难以维持下去,这必然会促使农业生产者加大农药化肥产品的使用,食品安全受到前所未有的挑战。这与人民日益增长的对高品质安全食品的需求产生了难以消除的矛盾。为了解决这一矛盾,《国民经济和社会发展第十三个五年规划纲要》提出要推进农业现代化,并指出“农业是全面建成小康社会和实现现代化的基础,必须加快转变农业发展方式,着力构建现代农业产业体系、生产体系、经营体系,提高农业质量效益和竞争力,走产出高效、产品安全、资源节约、环境友好的农业现代化道路”,着重提出要推进农业物联网应用,提高农业信息化、智能化和精准化水平。

农业现代化的一个重要体现就是实现现代化的设施农业\supercite{LiuLei2013}。温室作为设施农业的重要组成部分,可以减少自然环境对于农业生产的限制,合理高效利用生产资源,人为地创造适宜的农业生产环境,提高农作物产量和质量。但是目前国内温室大多停留在电气化及以下水平,依靠农业生产者的经验对温室环境进行人工控制或电气化控制。已经实现的温室自动化监控系统多为本地监控,农业生产人员需要到现场才能获取相关数据,不便于科学研究和生产应用\supercite{duodiandapeng}。随着科技的发展进步,农业物联网为农业生产环境监控提供了新思路,可以实现农业环境的远程监测和控制,从而实现农业管理的智能化、网络化、综合化、多样化,是实现农业现代化的关键技术之一\supercite{JiYuWuLianWang,WuLianWang,ZhongGuoSheShiNongYe}。

本文课题来源于《现代农业装备与设施的研发》(沪农科攻字(2009)第8-1号),上海市2009科技兴农重点攻关项目,由上海市农业科学委员会主导并开展工作。随着今年来物联网技术、网络技术、云服务技术和智能终端的快速发展,本文综合运用物联网技术、无线传感技术和多物理场仿真技术等多领域前沿技术,设计并实现了一套基于物联网和多物理场仿真的智能温室监测与控制系统,该系统可远程监测温室环境,并根据作物需求和当前环境实施精准的温室控制,从而科学高效的管理温室。
\section{国内外研究现状}
	\subsection{国内研究现状}
	我国是一个传统的农业大国,作为温室栽培的发源地,早在两千多年前就开始使用保护措施种植蔬菜。进入21世纪以来,我国的温室栽培技术得到快速提升,温室总面积不断增加,占据世界温室总面积的85\%以上,是温室栽培第一生产大国\supercite{WoGuoSheShiNongYe}。我国在现代温室方面的研究开始于20世纪30年代,日光温室首先在辽宁省应用于栽培蔬菜,随后我国从日本引进塑料薄膜技术开始中小拱棚的栽培,之后我国的温室一直处于发展缓慢的小规模低水平状态,直到上世纪70年代末,我国引进了一些国外的先进技术,我国的设施农业展开了新的篇章\supercite{xiandaiwenshi}。上世纪80年代开始,我国研究温室环境控制系统,并开始引入计算机控制,温室环境的控制效果得到了有效的改善\supercite{HanYi2016}。上世纪90年代后,我国在引进国外先进技术的基础上开始自主研制温室环境监控系统。上海交通大学、中国农业大学、中科院、农科院等科研院所和高校都研制出了具有不同特点的温室控制系统。
	
	经过多年的研究和时间,我国在现代温室方面的水平已经有了很大程度上的提高,但是在很多方面仍然落后于国际领先水平,主要体现在基础研究薄弱、设施结构不合理、装备和调控能力差、信息化和集约化程度低等方面\supercite{ZhangZhen2015,JiangWeijie2015} ,在温室环境的精准化、智能化控制等方面有较大的发展空间。
	\subsection{国外研究现状}
	美国、日本、荷兰等发达国家对于现代温室技术的研究起步较早,发展到今天,已经将计算机监控系统大规模应用到温室自动化生产中,在设备装备、环境控制和栽培技术等方面都处于国际领先水平。
	
	美国的温室主要为连栋玻璃温室,现代化水平非常高。美国是最早将计算机技术投入到温室生产管理中的国家,目前,美国约有82\%的温室采用计算机进行环境控制,27\%的农民还运用的网络技术\supercite{Kacira2011}。这样虽然提高了生产成本,但是也极大程度上给农业生产者带来了良好的经济效益\supercite{GuoShirong2012}。
\section{论文的主要内容与章节安排}
