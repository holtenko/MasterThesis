%# -*- coding: utf-8-unix -*-
%%==================================================
%% chapter0.tex for SJTU Master Thesis
%%==================================================


\chapter{绪论}
\label{chapter:Introduction}

\section{研究背景和意义}
自古以来,我们中国的祖先就知道“民以食为天”“粮食定,天下定”这一治国理政之道\supercite{ChenYunWenXuan},中国作为一个人口大国,解决全国人民的粮食食物问题是关系到国计民生的根本问题。根据农业科学院预测到2050年我国的人口将超过15亿,人口的大量增加也必然带来粮食需求量的猛增,要满足人口增长对粮食的需求,每年至少需要多生产1.2亿吨粮食。随着我国经济的飞速发展和人们生活水平的提高,人们不再仅仅关注于温饱的问题,更加追求品质生活,更加关注食品的安全、营养和多样\supercite{FengZhiming2007,ZhaoQiguo2011}。随之而来的,农业生产在不断扩大生产总量的同时也需要不断升级产业结构,以提供更加安全和高品质农副产品。

但是我国目前的农业生产的总体技术水平落后,现代化和信息化水平较低。农村的城镇化导致我国可用耕地面积不断减少,农村劳动力大量涌入城镇\supercite{YuJunli2001},务农人口急剧减少,农业生产人员逐步呈现老龄化和副业化趋势\supercite{LuoChaobin2005},传统的农业生产模式已经难以维持下去,这必然会促使农业生产者加大农药化肥产品的使用,食品安全受到前所未有的挑战,这与人民日益增长的对高品质安全食品的需求产生了难以消除的矛盾。同时随着中国市场的逐步开放,国内农产品市场与国际市场的竞争也在不断加剧。另一方面,我国还存在水资源紧缺、幅员辽阔但可用农业土地较少、生产能耗较高、基础设施建设不完善等问题。

为了解决这一系列问题,《国民经济和社会发展第十三个五年规划纲要》提出要推进农业现代化,并指出“农业是全面建成小康社会和实现现代化的基础,必须加快转变农业发展方式,着力构建现代农业产业体系、生产体系、经营体系,提高农业质量效益和竞争力,走产出高效、产品安全、资源节约、环境友好的农业现代化道路”,着重提出要推进农业物联网应用,提高农业信息化、智能化和精准化水平。

农业现代化的一个重要体现就是实现现代化的设施农业\supercite{LiuLei2013}。温室作为设施农业的重要组成部分,可以减少自然环境对于农业生产的限制,合理高效利用生产资源,人为地创造适宜的农业生产环境,提高农作物产量和质量,提高农业生产集约化程度。但是目前国内温室大多停留在电气化及以下水平,依靠农业生产者的经验对温室环境进行人工控制或电气化控制。已经实现的温室自动化监控系统多为本地监控,智能化和网络化程度较低,农业生产人员需要到现场才能获取相关数据,不便于生产应用和科学研究\supercite{duodiandapeng}。随着科技的发展进步,农业物联网为农业生产环境监控提供了新思路,通过物联网技术、传感器技术、互联网技术等先进的技术手段的综合运用可以实现农业环境的远程监测和控制,从而实现农业管理的智能化、网络化、综合化、多样化,是实现农业现代化的关键技术之一\supercite{JiYuWuLianWang,WuLianWang,ZhongGuoSheShiNongYe}。

本文课题来源于《现代农业装备与设施的研发》(沪农科攻字(2009)第8-1号),上海市2009科技兴农重点攻关项目,由上海市农业科学委员会主导并开展工作。随着今年来物联网技术、网络技术、云服务技术和智能终端的快速发展,本文综合运用物联网技术、无线传感技术和多物理场仿真技术等多领域前沿技术,设计并实现了一套基于物联网和计算流体力学仿真的智能温室监测与控制系统,该系统可远程监测和控制温室环境,让农业生产管理人员足不出户即可远程查看当前温室内的环境数据,通过视频查看当前温室内的实时情况,同时可通过远程操作温室内的作动器实现对温室内环境参数的控制。该系统还可结合种植作物的需求和当前温室内的实时环境根据智能化的温室自动控制策略实施精准的温室环境控制,从而科学高效的管理温室。消费者也可通过该系统了解到农作物的生产环境,一方面可以让消费者吃的安心放心,另一方面也可让消费者对农业生产进行监督,促进农业生产健康发展。
\section{国内外研究现状}
	\subsection{国内研究现状}
	我国是一个传统的农业大国,作为温室栽培的发源地,早在两千多年前就开始使用保护措施种植蔬菜。进入21世纪以来,我国的温室栽培技术得到快速提升,温室总面积不断增加,占据世界温室总面积的85\%以上,是温室栽培第一生产大国\supercite{WoGuoSheShiNongYe}。我国在现代温室方面的研究开始于20世纪30年代,日光温室首先在辽宁省应用于栽培蔬菜,随后我国从日本引进塑料薄膜技术开始中小拱棚的栽培,之后我国的温室一直处于发展缓慢的小规模低水平状态,直到上世纪70年代末,我国引进了一些国外的先进技术,我国的设施农业展开了新的篇章\supercite{xiandaiwenshi}。上世纪80年代开始,我国研究温室环境控制系统,并开始引入计算机控制,温室环境的控制效果得到了有效的改善\supercite{HanYi2016}。上世纪90年代后,我国在引进国外先进技术的基础上开始自主研制温室环境监控系统。上海交通大学、中国农业大学、中科院、农科院等科研院所和高校都研制出了具有不同特点的温室控制系统。
	
	经过多年的研究和时间,我国在现代温室方面的水平已经有了很大程度上的提高,但是在很多方面仍然落后于国际领先水平,主要体现在基础研究薄弱、设施结构不合理、装备和调控能力差、信息化和集约化程度低等方面\supercite{ZhangZhen2015,JiangWeijie2015} ,在温室环境的精准化、智能化控制等方面有较大的发展空间。
	\subsection{国外研究现状}
	美国、日本、荷兰等发达国家对于现代温室技术的研究起步较早,发展到今天,已经将计算机监控系统大规模应用到温室自动化生产中,在设备装备、环境控制和栽培技术等方面都处于国际领先水平。
	
	西欧是世界现代温室的发源地,虽然西欧国家的设施农业面积在世界世界设施农业总面积中占比不高,但是其现代化水平、生产质量和生产能力都非常高\supercite{GuoShirong2012}。荷兰是其中比较典型的设施农业非常发达的国家之一,主要对花卉和蔬菜进行专业化集约化生产,花卉和蔬菜出口量均居世界第一,有“欧洲菜园子”之称,其温室主要以玻璃温室为主,约占世界玻璃温室总面积的26\%以上\supercite{Watson,JiHong2007},农业生产的自动化程度很高,生产效率很高位居世界第一\supercite{QinLiu2015},其温室配套设备在满足内需的同时还大量出口到国外\supercite{Tavoletti2008Cutting}。
	
	美国的温室主要为连栋玻璃温室,现代化水平非常高。美国是最早将计算机技术投入到温室生产管理中的国家,通过政府的大力支持,采取了一系列的优惠政策,为农业现代化和信息化的发展创造了良好的氛围和环境。目前,美国约有82\%的温室采用计算机进行环境控制,27\%的农民还运用的网络技术\supercite{Kacira2011}。这样虽然提高了生产成本,但是也极大程度上给农业生产者带来了良好的经济效益\supercite{GuoShirong2012}。
	
	以色列是典型的中东国家,气候干旱,水资源匮乏,但是以色列的设施农业却非常发达,创造了沙漠中的片片绿洲,其开发的节水灌溉系统引入了计算机控制,可以精准控制水肥比例,处于世界领先水平,此外以色列对于作物生长机理的研究也比较深入,可据此建立合理的温室控制策略\supercite{GuoShirong2012,TangLibiao2003},主要生产高质量的花卉和果蔬,不仅可以自给自足还大量出口国外,享有”欧洲厨房“的美誉\supercite{QinLiu2015}。
	
	日本是一个资源非常匮乏的国家,且人口密度非常大,因此日本大力发展集约化、自动化的设施农业,主要生产果蔬和花卉,且依赖程度非常高\supercite{YangChunjun2010}。日本是最先提出植物工厂的概念,突破了土地和环境的限制,利用自动控制技术、电子技术、生物技术、机器人和新材料可以实现作物的全年连续生产,对于解决全球粮食问题和环境问题有着重要的现实意义\supercite{HuYongguang2002}。
	
	由此可见,发达国家在设施农业和现代化温室方面的研究处于非常明显的领先地位,在温室管理中大面积采用计算机控制系统,自动化和集约化程度都较高,目前正在向更为先进的智能化、精准化和网络化的方向发展。
\section{论文的主要内容与章节安排}
	\subsection{研究内容和创新点}
	本文结合物联网技术和计算力体力学仿真技术,对温室智能化监测与控制系统展开研究,主要有以下五部分内容:
		\begin{enumerate}
  			\item 适用于智能温室的农业物联网架构方案设计提出。
  			\item 针对温室机械通风三维全尺度瞬态及稳态计算流体力学仿真模型。
  			\item 针对南方地区典型的连栋塑料温室夏季机械通风实验与温室仿真模型验证。
  			\item 基于农业物联网的智能温室监测与控制系统的软件与硬件设计和实现。
  			\item 基于温室机械通风仿真模型的传感器测点位置选择与优化,及机械通风控制策略优化设计。
		\end{enumerate}
	相比于已有的相关研究,本文具有如下创新点:
		\begin{enumerate}
  			\item 本文根据农业生产环境的特殊性,通过分析温室监测与控制的特殊需求并参考物联网的标准框架,设计提出了一种适用于智能温室的农业物联网架构方案。设计并实现了基于该架构方案的智能温室远程监测与控制系统,并应用与连栋塑料温室,实现了对温室的智能感知、远程控制和智能控制,具有可靠性高、监测范围广、可灵活扩展、运行功耗低的优点,满足温室的智能运行和科学管理需求。
  			\item 为了解决我国南方地区夏季长期高温的恶劣天气对温室作物生长的严重影响,提高降温效果且减少通风能耗,本文以南方地区典型的连栋塑料温室为研究对象,针对温室机械通风,建立了三维全尺度瞬态及稳态计算流体力学仿真模型,研究了机械通风条件下温室内的气温变化和分布规律。通过在本文智能温室监测与控制系统平台上进行机械通风实验,验证了仿真模型瞬态和稳态计算的准确性和有效性。通过仿真模型模拟了室外高温条件下的风机数量、温室长度、入口温度和环境温度变化等参数对机械通风降温效果的影响程度,并模拟了不同数量风机启闭控制的降温效果,为智能温室监测和控制系统提供了优化的夏季机械通风控制策略,同时提供了优化的传感器测点布置策略,实现了以少量传感器测点数据准确反映大型温室气温分布。
		\end{enumerate}	
	\subsection{章节安排}
	本文的章节结构安排如下:
	
	第一章,绪论。介绍了本文课题的研究背景和研究意义、国内外在设施农业和现代化温室方面的研究进展和发展趋势、本文的主要研究内容和章节安排。
	
	第二章,基于农业物联网的智能温室架构。介绍物联网,提出基于农业物联网的智能温室四层架构方案,并对每层结构进行阐述和设计。
	
	第三章,智能温室监测与控制系统。详细介绍了基于本文农业物联网智能温室架构方案结合实际应用的具体实现,包括各层的硬件和软件实现。
	
	第四章,温室计算流体力学仿真及验证。基于计算流体力学建立三维全尺度瞬态及稳态机械通风仿真模型,利用本文智能温室监测与控制系统进行机械通风实验,并通过实验结果对仿真模型进行了验证。
	
	第五章,智能温室的实践与应用。详细介绍本文所设计实现的智能温室远程监测与控制系统在温室现场的实践与应用,以及温室仿真模型在实际应用中对于传感器测点位置选择优化作用和对温室机械通风控制策略的优化设计。
	
	第六章,总结与展望。总结研究内容,展望研究内容的发展前景和改进方向。