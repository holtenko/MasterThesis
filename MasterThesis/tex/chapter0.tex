%# -*- coding: utf-8-unix -*-
%%==================================================
%% chapter0.tex for SJTU Master Thesis
%%==================================================


\chapter{绪论}
\label{chapter:Introduction}

\section{研究背景}
自古以来,我们中国的祖先就知道“民以食为天”“粮食定,天下定”这一治国理政之道\supercite{ChenYunWenXuan},中国作为一个人口大国,解决全国人民的粮食食物问题是关系到国计民生的根本问题。根据农业科学院预测到2050年我国的人口将超过15亿,人口的增加也必然带来粮食需求量的增加。随着我国经济的飞速发展和人们生活水平的提高,人们不再仅仅关注于温饱的问题,更加追求品质生活,更加关注食品的安全、营养和多样\supercite{FengZhiming2007,ZhaoQiguo2011}。随之而来的,农业生产在不断扩大总量的同时也需要不断升级产业结构,以提供更加安全和高品质农副产品。但是我国目前的农业生产的总体技术水平落后,现代化和信息化水平较低。农村的城镇化导致我国可用耕地面积不断减少,农村劳动力大量涌入城镇\supercite{YuJunli2001},务农人口急剧减少,农业生产人员逐步呈现老龄化和副业化\supercite{LuoChaobin2005},传统的农业生产模式已经难以维持下去,这必然会促使农业生产者加大农药化肥产品的使用,食品安全受到前所未有的挑战。这与人民日益增长的对高品质安全食品的需求产生了难以消除的矛盾。为了解决这一矛盾,《国民经济和社会发展第十三个五年规划纲要》提出要推进农业现代化,并指出“农业是全面建成小康社会和实现现代化的基础,必须加快转变农业发展方式,着力构建现代农业产业体系、生产体系、经营体系,提高农业质量效益和竞争力,走产出高效、产品安全、资源节约、环境友好的农业现代化道路”,着重提出要推进农业物联网应用,提高农业信息化、智能化和精准化水平。

农业现代化的一个重要体现就是实现现代化的设施农业。温室作为设施农业的重要组成部分,可以减少自然环境对于农业生产的限制,人为地为农作物创造适宜的生长环境,
\section{研究目的和意义}

\section{国内外研究现状}

\section{论文的主要内容与章节安排}
